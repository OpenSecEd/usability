% Since this a solution template for a generic talk, very little can
% be said about how it should be structured. However, the talk length
% of between 15min and 45min and the theme suggest that you stick to
% the following rules:  

% - Exactly two or three sections (other than the summary).
% - At *most* three subsections per section.
% - Talk about 30s to 2min per frame. So there should be between about
%   15 and 30 frames, all told.


% XXX have psychology parts connect more to security
\section{Social engineering}

\subsection{What's social engineering?}

\begin{frame}
  \begin{figure}
    \includegraphics[height=0.7\textheight]{security.png}
    \caption{\enquote{Only amateurs attack machines; professionals target 
        people.} [Bruce Schneier].
    Image:~\cite{xkcd538}.}
  \end{figure}
\end{frame}

\begin{frame}
  \begin{example}[Socially enginnering Helpdesk]
    \begin{itemize}
      \item Two students changed my password by calling Helpdesk.
    \end{itemize}

    \pause

    \begin{enumerate}
      \item Asked a teacher to book a room for them to work on a project.
      \item Researched my background information.
      \item Called Helpdesk from the booked room --- internal telephone!
        \begin{itemize}
          \item Said they were going to do teaching.
          \item The lecture was supposed to start a few minutes ago.
          \item But they couldn't log in \dots
        \end{itemize}
      \item Helpdesk reset the password.
    \end{enumerate}
  \end{example}
\end{frame}

\begin{frame}
  \begin{remark}
    \begin{itemize}
      \item Well-researched, well-practiced.
      \item Created a credible, stressful situation.
    \end{itemize}
  \end{remark}
\end{frame}

\subsection{Is it a problem?}

\begin{frame}
  \begin{block}{Our naturally developed protection}
    \begin{itemize}
      \item Our natural protection has developed over millions of years.
      \item It's based on \enquote{here and now}.
      \item The threat has changed context during the past decades.
      \item Evolution isn't that fast.
    \end{itemize}
  \end{block}
\end{frame}

\begin{frame}
  \begin{definition}[Pretexting]
    \begin{itemize}
      \item To call someone who has access to information and pretend to be 
        authorized to know it.
    \end{itemize}
  \end{definition}

  \pause

  \begin{example}
    \begin{itemize}
      \item \Eg pretend to be a physician in another hospital, have an urgency 
        and need some information from the patient records.
    \end{itemize}
  \end{example}
\end{frame}

\begin{frame}
  \begin{example}
    \begin{itemize}
      \item Study in the UK in 1996~\cite{Anderson2008sea}.
      \item Trained staff to spot pretexting attempts.
      \item Discovered 30 false call per week \dots
    \end{itemize}
  \end{example}
\end{frame}

\begin{frame}
  \begin{example}
    \begin{itemize}
      \item A false press release saying that the CEO had resigned and the 
        profits would be recalculated~\cite{Anderson2008sea}.
      \item The company stock dropped more than \SI{60}{\%}.
    \end{itemize}
  \end{example}
\end{frame}

\begin{frame}
  \begin{example}
    \begin{itemize}
      \item There was an audit at IRS in the US in 2007.
      \item They called 102 users across the organization.
      \item They asked them to change their password to a given value.
      \item 62 of them followed the instruction.
    \end{itemize}
  \end{example}
\end{frame}

\begin{frame}
  \begin{remark}
    \begin{itemize}
      \item This type of attack is \emph{social engineering}.
    \end{itemize}
  \end{remark}
\end{frame}

\subsection{Phishing}

\begin{frame}
  \begin{remark}
    \begin{itemize}
      \item Military organizations have always targeted each other like this.
      \item They trained their staff and kept alertness thorough attacks.

        \pause

      \item Civilian organizations, \eg banks, can also train their staff.

        \pause

      \item However, it's difficult to train all their customers.
      \item This must be solved through usability.
    \end{itemize}
  \end{remark}
\end{frame}

\begin{frame}
  \begin{question}
    \begin{itemize}
      \item How do you know the card terminal in the supermarket is legit?
      \item How to you know if it's a phishing URL or marketing using a service 
        to track clicks?
    \end{itemize}
  \end{question}

  \begin{remark}
    \begin{itemize}
      \item Probably you don't, like everyone else.
    \end{itemize}
  \end{remark}
\end{frame}

\begin{frame}
  \begin{summary}
    \begin{itemize}
      \item Technology makes it easier to attack customers.
      \item The attackers reuse real emails, change URLs.
      \item The attackers want data: usernames, passwords, personal information 
        for social engineering attacks.
    \end{itemize}
  \end{summary}

  \begin{remark}
    \begin{itemize}
      \item Try Google's phishing quiz: 
        \url{https://phishingquiz.withgoogle.com/}.
    \end{itemize}
  \end{remark}
\end{frame}

\subsection{A more elaborate example}

\begin{frame}
  \begin{example}[Mat Honan~\cite{MatHonan}]
    \begin{itemize}
      \item Attackers took over and removed his Google account.
      \item They took over his Twitter account and posted \enquote{unfriendly} 
        comments.
      \item They took over his AppleID account and removed all data from all 
        his Apple devices.
    \end{itemize}
  \end{example}
\end{frame}

\begin{frame}
  \begin{example}[Mat Honan, how?~cite{MatHonan}]
    \begin{itemize}
      \item Amazon and AppleCare had incompatible security.
      \item An unfortunate connection between email addresses (me.com, Gmail).
    \end{itemize}
  \end{example}
\end{frame}


%%%%%%%%%%%%%%%%%%%%%%

\begin{frame}[allowframebreaks]
	\small
  \printbibliography
\end{frame}

