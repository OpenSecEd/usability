% $Id$
%\documentclass[handout]{beamer}
\documentclass{beamer}
\usepackage[utf8]{inputenc}
\usepackage[T1]{fontenc}
\usepackage[swedish,british]{babel}
\usepackage{url}
\usepackage{graphicx}
\usepackage{color}
\usepackage{subfig}
\usepackage{multicol}
\usepackage{booktabs}
\usepackage[binary,squaren]{SIunits}
\usepackage{csquotes}

\setbeamertemplate{bibliography item}[text]
\usepackage[natbib,style=alphabetic,maxbibnames=99,backend=bibtex8]{biblatex}
\addbibresource{overview.bib}

\mode<presentation>{%
  \usetheme{Berlin}
  \setbeamercovered{transparent}
}
%\setbeamertemplate{footline}{\insertframenumber}

\title{%
  Usable Security: An Overview
}
\author{Daniel Bosk}
\institute[MIUN IKS]{%
  Department of Information and Communication Systems,\\
  Mid Sweden University, SE-851\,70 Sundsvall.
}
\date{\today}

%\pgfdeclareimage[height=0.65cm]{university-logo}{MU_logotyp_int_CMYK.pdf}
%\logo{\pgfuseimage{university-logo}}

\AtBeginSection[]{%
  \begin{frame}<beamer>
    \small
    \tableofcontents[currentsection]
  \end{frame}
}

\begin{document}

\begin{frame}
  \titlepage{}
\end{frame}

\begin{frame}
  \begin{quote}
    Humans are incapable of securely storing high-quality cryptographic keys, 
    and they have unacceptable speed and accuracy when performing cryptographic 
    operations.
    (They are also large, expensive to maintain, difficult to manage, and they 
    pollute the environment.
    It is astonishing that these devices continue to be manufactured and 
    deployed.
    But they are sufficiently pervasive that we must design our protocols 
    around their limitations.)
  \end{quote}
  \begin{flushright}
    \small
    Kaufmann, Perlman and Speciner
  \end{flushright}
\end{frame}


% Since this a solution template for a generic talk, very little can
% be said about how it should be structured. However, the talk length
% of between 15min and 45min and the theme suggest that you stick to
% the following rules:  

% - Exactly two or three sections (other than the summary).
% - At *most* three subsections per section.
% - Talk about 30s to 2min per frame. So there should be between about
%   15 and 30 frames, all told.


% XXX have psychology parts connect more to security
\section{Social engineering}

\subsection{What's social engineering?}

\begin{frame}
  \begin{figure}
    \includegraphics[height=0.7\textheight]{security.png}
    \caption{''Only amateurs attack machines; professionals target people.'' 
    (Bruce Schnier).
    Bild:~\cite{xkcd538}.}
  \end{figure}
\end{frame}

\begin{frame}
  \begin{itemize}
    \item We figure out the weaknesses in human psychology.
    \item Then we exploit these to our advantage.
  \end{itemize}
\end{frame}

\subsection{Is it a problem?}

\begin{frame}{Vårt naturligt utvecklade skydd}
  \begin{itemize}
    \item Vårt naturliga skydd som utvecklats under miljontals år är baserat på 
      \enquote{här och nu}.

    \item Hotet har bytt kontext under de senaste decennierna.

    \item Evolutionen är betydligt långsammare \dots

  \end{itemize}
\end{frame}

\begin{frame}
  \begin{block}{Pretexting}
    \begin{itemize}
      \item Att ringa någon som har tillgång till informationen och låtsas vara 
        behörig att få veta.

      \item Exempelvis att låtsas vara behandlande läkare av en patient i en akut 
        situation för att få ut information ur journalen.
    \end{itemize}
  \end{block}
\end{frame}

\begin{frame}
  \begin{example}
    \begin{itemize}
      \item Undersökning genomfördes i UK 1996~\cite{Anderson2008sea}.
      \item Utbildade personalen vid vårdinrättning om pretextingattacker.
      \item Upptäckte 30 falska samtal i veckan.
    \end{itemize}
  \end{example}
\end{frame}

\begin{frame}
  \begin{example}
    \begin{itemize}
      \item Från verkligheten~\cite{Anderson2008sea}: Ett falskt pressmeddelande 
        publicerades som sade att VD avgått och att vinsten skulle räknas om.
        \begin{itemize}
          \item Aktien föll med över \unit{60}{\%} innan det uppdagades.
        \end{itemize}

      \item Generellt går denna typ attack under \emph{social engineering}.

    \end{itemize}
  \end{example}
\end{frame}

\begin{frame}
  \begin{example}
    \begin{itemize}
      \item Vid granskning av IRS 2007 ringdes 102 personer spridda över hela 
        organisationen upp.

      \item De ombads uppge sitt användarnamn och ändra sitt lösenord till ett 
        givet värde.

      \item 62 av dem följde instruktionen.

    \end{itemize}
  \end{example}
\end{frame}

\begin{frame}
  \begin{example}
    Användare säljer sina lösenord för en chokladkaka~\cite{dn2010choklad}.
  \end{example}

  \begin{example}
    \begin{itemize}
      \item Personer tar främmande USB-minnen och använder dem med sina datorer.
        \begin{itemize}
            % XXX lookup ref for 46% of people picking up usb sticks
          \item Närmare bestämt \unit{46}{\%} av ekonomicheferna vid 500 
            börsnoterade företag~\cite{pickupusb}.
          \item \unit{66}{\%} innehåller sabotageprogram~\cite{Sophos2011usbmal}.
        \end{itemize}

    \end{itemize}
  \end{example}
\end{frame}

\subsection{Phishing}

\begin{frame}
  \begin{itemize}
    \item Militära organisationer har alltid haft varandras personal som 
      måltavla för denna typer av attacker.

    \item De har fördelen att kunna utbilda sin personal.

    \item Vanliga organisationer kan också utbilda sin personal.

    \item Det blir desto svårare att utbilda kunder eller andra personer som 
      berör verksamheten men inte är en del av organisationen.

    \item Så detta måste lösas i gränssnittet.

  \end{itemize}
\end{frame}

\begin{frame}
  \begin{itemize}
    \item Det är inte längre organisationen som angrips utan kunder och 
      personer runt omkring.

    \item Angripare återanvänder riktiga e-brev med utbytta URL:er.

    \item Vill ha ut användarnamn, lösenord, personuppgifter, \dots
  \end{itemize}

  \begin{block}{Råd}
    \begin{itemize}
      \item Klicka aldrig på URL:er som skickats till dig.
      \item Skicka aldrig klickbara \enquote{Klicka här}-URL:er till någon.
    \end{itemize}
  \end{block}

\end{frame}

\subsection{An example}

\begin{frame}{Mat Honan of Wired Magazine}
  \begin{example}[Vad?]
    \begin{itemize}
      \item Angripare tog över och tog bort Googlekonto.
      \item Tog över Twitterkonto och postade kränkande kommentarer.
      \item Tog över AppleID-konto och tog bort alla data från alla 
        Apple-enheter.
    \end{itemize}
  \end{example}
\end{frame}

\begin{frame}{Mat Honan of Wired Magazine}
  \begin{example}[Hur?]
    \begin{itemize}
      \item Brister hos Amazon.
      \item Brister hos AppleCare.
      \item Olycklig koppling av e-postadresser för Me.com och Gmail.
      \item Detta gav dem även Twitter.
    \end{itemize}
  \end{example}
\end{frame}


\section{How humans function}

\subsection{Grundläggande psykologi}

\begin{frame}
  \begin{block}{Mental models}
    \begin{itemize}
      \item Mentala modeller låter oss identifiera människor, ljud, 
        \enquote{koncept} bättre än datorer.

      \item Dock gör oss även sårbara när fel modell aktiveras eller modellen 
        inte är i linje med verkligheten.
    \end{itemize}
  \end{block}

  \pause{}

  \begin{example}[Säker anslutning]
    \begin{itemize}
      \item Användaren förstår inte SSL/TLS\@.
      \item Hänglåset betyder säkerhet.
      \item En phishingsida har ett signerat certifikat.
    \end{itemize}
  \end{example}
\end{frame}

\begin{frame}
  \begin{block}{Capture errors}
    Ett inövat beteende används istället för korrekt.
  \end{block}

  \pause{}

  \begin{example}
    \begin{itemize}
      \item Svänger ut på motorvägen mot Sundsvall istället för Härnösand.
      \item Åker hem istället för till affären efter jobbet.
      \item Klickar \enquote{automatiskt} på OK-knappen utan att tänka efter.
    \end{itemize}
  \end{example}
\end{frame}

\begin{frame}
  \begin{block}{Post-completion error}
    När målet är nått är uppgiften genomförd, eller \dots
  \end{block}

  \pause{}

  \begin{example}
    \begin{itemize}
      \item Uttagsautomater som ger pengarna före kortet gör att fler glömmer 
        kortet i automaten.
    \end{itemize}
  \end{example}
\end{frame}

\begin{frame}
  \begin{figure}
    \includegraphics[height=0.7\textheight]{citations.png}
    \caption{En kort seriestrip om hur vi är okritiska mot text som, till 
    synes, har referenser.
    Bild:~\cite{xkcd906}.}
  \end{figure}
\end{frame}

\begin{frame}
  \begin{block}{Cognitive load}
    \begin{itemize}
      \item Kognitiv belastning påverkar oss hårt.
      \item Handlingar som följer någon form av regel.
      \item Vid hög kognitiv belastning kan fel regel följas, exempelvis 
        starkaste regeln istället för lämpligaste.
    \end{itemize}
  \end{block}
  
  \pause{}

  \begin{example}
    \begin{itemize}
      \item Det är säkert då det står \enquote{https} i URL:en, eller ikonen 
        med hänglåset.
      \item Att hitta bankens namn är en starkare regel än att tänka på dess 
        position; \url{https://www.swedbank.se.fraudulentbanks.com}.
    \end{itemize}
  \end{example}
\end{frame}

\subsection{Biases}

\begin{frame}
  \begin{block}{Automation bias}
    We trust the computer to have done the work properly, so we relax.
  \end{block}
  \begin{example}
    \begin{itemize}
      \item Två studier visar att tilliten till resultaten från Google är stor.

      \item Studenter valde länkar högre upp i träfflistan trots att 
        sammanfattningarna var mindre relevanta än träffar längre 
        ned~\cite{Pan2007igw}.

      \item I en nyligare genomförd studie~\cite{Epstein2013dar} visas att 
        sökresultat kan förändra personers röstningspreferenser utan att 
        uppmärksammas.

      \item Detta gör sökmotoroptimering till ett farligt område.

    \end{itemize}
  \end{example}
\end{frame}

\begin{frame}
  \begin{block}{Confirmation bias}
    We seek out information that confirms our beliefs.
  \end{block}

  \pause{}

  \begin{example}
    \begin{itemize}
      \item This makes us bad at testing hypotheses.
      \item Test if the site is a phishing site by giving it username and 
        password.
      \item If the site knows them, it must be legit.
    \end{itemize}
  \end{example}

  \pause{}

  \begin{block}{Reverse-authorization fallacy}
    Give me the username and password and I'll verify them.
  \end{block}
\end{frame}

\begin{frame}
  \begin{block}{Disconfirmation bias}
    We rather accept plausible but wrong, than implausible but correct.
  \end{block}

  \pause{}

  \begin{example}
    \begin{itemize}
      \item If the website behaves as the bank \dots
      \item Then it must be the bank \dots
      \item Although it's run from Russia.
    \end{itemize}
  \end{example}
\end{frame}

\begin{frame}
  \begin{block}{Projection bias}
    Everyone thinks like me.
  \end{block}

  \begin{example}
    \begin{itemize}
      \item Someone designs a system and expects the users to think like the 
        designer.
      \item If you're logged in, then you must be a good guy like me.
    \end{itemize}
  \end{example}
\end{frame}

\begin{frame}
  \begin{block}{Blind-spot bias}
    Biases are unconscious, that makes them difficult to see for conscious 
    thought.
  \end{block}
\end{frame}

%\subsection{How me make decisions}
%
%\begin{frame}
%  \begin{itemize}
%    \item Heuristiker människor använder för beslut ligger på gränsen mellan 
%      rationellt tänkande och direkta sinnesintryck.
%
%    \item Risker: I många fall tycker vi mindre om att förlora \unit{100}{kr} 
%      vi redan har än att vinna \unit{100}{kr} vi inte har.
%      \begin{itemize}
%        \item Marknadsföring: ''spara \unit{100}{kr}''.
%      \end{itemize}
%
%    \item Vi är dåliga på att uppskatta sannolikheter:
%      \begin{itemize}
%        \item Baserar härledningar på enkla analogier.
%        \item Tillgänglighetsheuristiken: lättillgängliga data har större vikt 
%          vid resonemang.
%        \item Och vi jämför med nyliga händelser.
%        \item Förankringseffekten: vi gör en initial uppskattning och 
%          förbättrar vid behov.
%      \end{itemize}
%
%  \end{itemize}
%\end{frame}
%
%\begin{frame}
%  \begin{itemize}
%    \item Vi är mer benägna att vara skeptiska mot något vi hört än något vi 
%      sett.
%    \item Överskattar risken för terroristattacker jämfört med bilolyckor.
%    \item Beror dels på synligheten i media.
%    \item Gilbert: ''If only gay sex caused global warming''.
%    \item Global uppvärmning bryter inte mot någons (religiösa?) värderingar, 
%      är långtgående hot.
%    \item Vi är anpassade för snabba förändringar i omvärlden och tydliga akuta 
%      hot.
%    \item Vi är också mindre rädda när vi (tror) att vi har kontroll, 
%      exempelvis att köra bilen jämfört med att sitta som passagerare.
%    \item Vi är också riskaversiva: ''kvitt eller dubbelt'' går inte hem.
%  \end{itemize}
%\end{frame}
%
%\begin{frame}
%  \begin{itemize}
%    \item Kan dela upp psyket i kognitivt och affektivt system.
%
%    \item Utvecklingsbiologin har sett att olika processer används för sociala 
%      och fysikaliska fenomen.
%
%    \item Barn försöker att förklara händelser med sin förståelse för fysik, 
%      när den inte räcker försöker de förklara med avsiktliga handlingar (det 
%      affektiva tar vid).
%
%    \item Detta leder till att de tar hjälp av vuxna.
%
%  \end{itemize}
%\end{frame}
%
%\begin{frame}{Bieffekter}
%  \begin{itemize}
%    \item Människor försöker att förklara händelser med intention snarare än 
%      situation.
%
%    \item Vidare leder detta till att om vi får den affektiva sidan att ta över 
%      är personen mindre uppmärksam och sämre att uppskatta sannolikheter.
%
%  \end{itemize}
%\end{frame}

\subsection{Socialpsykologi}

\begin{frame}
  \begin{itemize}
    \item Det sociala samspelet har stor inverkan på individen.

    \item 1951 visades att en individ kunde bortse från uppenbara bevis bara 
      för att följa gruppen.

      \pause{}

    \item Vidare har visats att individer kan göra helt moralvidriga saker 
      under order från en auktoritet, Officer Scott 1995--2005:
      \begin{itemize}
        \item Ringde upp restaurangchefer och låtsades vara polis.
        \item Tvingade fram strippsökningar av oskyldiga unga anställda.
      \end{itemize}

      \pause{}

    \item Detta kan ske även utan order från en auktoritet, Stanford Prisoner 
      Experiment 1971:
      \begin{itemize}
        \item 12 vakter, 12 fångar.
        \item Vakterna blev snabbt sadistiska auktoriteter.
      \end{itemize}

  \end{itemize}
\end{frame}

\begin{frame}
  \begin{itemize}
    \item Sociala medier?

      \pause{}

    \item Det ser ut som att \enquote{alla andra} har gjort det.
    \item Exempelvis bedrägerier som sprids via Facebook.
    \item Oftast är de orsakade av sabotageprogram.
  \end{itemize}
\end{frame}

%\begin{frame}{Kognitiv dissonansteori}
%  \begin{itemize}
%    \item Människor tycker inte om motstridigheter.
%
%    \item Söker information för att bekräfta tidigare kunskap eller mentala 
%      modeller.
%
%    \item Information som motstrider bortses ifrån, vill inte tro att man 
%      tidigare haft fel eller bryta från alla andra.
%
%  \end{itemize}
%\end{frame}


\section{Authentication}

\subsection{Lösenord}

\begin{frame}{Användbarhet?}
  \begin{itemize}
    \item Svårt att komma ihåg detaljer som används sällan.
    \item Svårt att komma ihåg detaljer som ändras ofta.
    \item Svårt att komma ihåg och särskilja många liknande detaljer.
    \item Svårt att minnas ord utan betydelse.
    \item Kan ej glömma på begäran.
    \item Att minnas är svårare än att känna igen.
  \end{itemize}
\end{frame}

\begin{frame}
  \begin{itemize}
    \item Enklare att komma ihåg saker som används ofta.
    \item Enklare att minnas saker i kontext.
    \item Men \dots
  \end{itemize}
\end{frame}

\begin{frame}
  \begin{figure}
    \includegraphics[height=0.65\textheight]{password_strength.png}
    \caption{%
      Hard to remember, easy to guess.
      Easy to remember, hard to guess.
      Bild:~\cite{xkcd936}.
    }
  \end{figure}
\end{frame}

\subsection{Alternatives}

\begin{frame}
  \begin{block}{Simson Garfinkel:}
    \begin{itemize}
      \item Something you had once
      \item Something you've forgotten
      \item Something you once were.
    \end{itemize}
  \end{block}
\end{frame}

\begin{frame}
  \begin{block}{Really?}
    \begin{description}
      \item[Vet] Lösenord.
      \item[Har] Koddosa, som oftast skyddas av ett lösenord.
      \item[Är] Fingeravtryck, som oftast kombineras med ett lösenord.
    \end{description}
  \end{block}
\end{frame}

\begin{frame}{Komplexiteten hos lösenord}
  \begin{itemize}
    \item PIN-koden för betalkortet, har endast tre försök sedan slutar kortet 
      att fungera.

      \pause{}

    \item Lösenordet för webbmailen, vore väldigt jobbigt om den blev låst.
      Hur låsa upp?

      \pause{}

    \item Krypterat data, har ej kontroll över antal försök.

  \end{itemize}
\end{frame}

\subsection{Workarounds}

\begin{frame}
  \begin{example}[Other types of passwords]
    \begin{itemize}
      \item Personnummer (även användarnamn).
      \item Kortnummer, medlemsnummer.
      \item Husdjurets namn.
      \item \enquote{Mother's maiden name}.
    \end{itemize}
  \end{example}
\end{frame}

\begin{frame}
  \begin{figure}
    \includegraphics[height=0.65\textheight]{pet_security_question.png}
    \caption{En seriestrip som antyder det bisarra med säkerhetsfrågor.
    Namnge dina husdjur med omsorg, du kommer att använda deras namn som 
    säkerhetsfråga resten av livet.}
  \end{figure}
\end{frame}

\subsection{Problems to solve}

\begin{frame}
  \begin{enumerate}
    \item Kommer användaren att mata in rätt lösenord tillräckligt ofta?

    \item Kan användaren minnas lösenordet, eller kommer denne att skriva ner 
      det på en lapp?
      Väljer användaren ett lösenord som är lätt att gissa?

  \end{enumerate}
\end{frame}

\begin{frame}
  \begin{example}[Entering passwords]
    \begin{itemize}
      \item Muntligen ange ett nummer: hotel-, biljettbokningar, hämta ut paket 
        från Posten.

      \item Mata in långa sifferkombinationer: mjukvarulicenser, refillkort, 
        OCR-nummer för räkningar.

      \item Att skriva dem i grupper om tre till fyra underlättar avsevärt.

      \item Längre lösenord, större sannolikhet att skriva fel.

    \end{itemize}
  \end{example}
\end{frame}

\begin{frame}
  \begin{example}[Remembering passwords]
    \begin{itemize}
      \item Välj ett lösenord du inte kan minnas och skriv inte ner det.

      \item xkcd:s \enquote{correct horse battery staple}, enkelt att komma 
        ihåg men svårare att skriva.

      \item Men om man bara behöver skriva det sällan, då är det mindre problem.

    \end{itemize}
  \end{example}
\end{frame}

\begin{frame}
  \begin{itemize}
    \item \citet{Komanduri2011opa} gjorde en undersökning om säkerhet och 
      användbarhet hos olika lösenordspolicyer.

      \pause{}

    \item Hade följande olika policyer:
      \begin{description}
        \item[basic8] Minst åtta tecken.

        \item[dictionary8] Minst åtta tecken, får inte finns med i ordlistan.

        \item[comprehensive8] Minst åtta tecken, måste innehålla små och stora 
          bokstäver, samt siffror och specialtecken.

        \item[basic16] Minst 16 tecken.
      \end{description}

      \pause{}

    \item Säkerheten var bäst hos basic16 (högst entropi), comprehensive8 var 
      näst bäst.

    \item Användbarhetsmässigt var basic16 bäst: användarna hade färre problem 
      att skriva in lösenordet och att komma ihåg det.
  \end{itemize}
\end{frame}

\begin{frame}
  \begin{example}[Real passwords]
    \begin{columns}
      \begin{column}{0.4\textwidth}
        \par
        Från~\cite{Oberheide2010bao}:
        \begin{itemize}
          \item 123456
          \item password
          \item 12345678
          \item qwerty
          \item abc123
        \end{itemize}
      \end{column}
      \begin{column}{0.4\textwidth}
        \par
        Från~\cite{Cluley2012twp}:
        \begin{itemize}
          \item 123456
          \item password
          \item welcome
          \item ninja
          \item abc123
        \end{itemize}
      \end{column}
    \end{columns}
  \end{example}
\end{frame}

% XXX move password cracking numbers to infotheory lecture
%\begin{frame}{Attackera ett konto eller alla}
%  \begin{block}{Ett givet konto}
%    \begin{itemize}
%      \item Svårt med lösenordsknäckning, måste testa \(|P|/2\) där \(P\) är 
%        mängden av alla lösenord.
%      \item Med phishing kallas detta \emph{spear phishing}.
%    \end{itemize}
%  \end{block}
%  \begin{block}{Alla konton på ett system}
%    \begin{itemize}
%      \item Avsevärt mycket enkare, närmare \(|P|/|U|\) där \(U\) är mängden av 
%        användare.
%      \item Kan använda phishing, räcker med att en användare faller för det.
%    \end{itemize}
%  \end{block}
%  \begin{block}{Alla konton på alla system}
%    \begin{itemize}
%      \item Knäck lösenord för ett enkelt system.
%      \item Phishing för ett system med dåliga policyer.
%      \item Sannolikt återanvänds lösenord i andra system.
%    \end{itemize}
%  \end{block}
%\end{frame}

\begin{frame}
  \begin{block}{Trusted path}
    Vi måste veta om vi kan lita på kommunikationskanalen.
  \end{block}

  \pause{}

  \begin{example}
    \begin{itemize}
      \item Är det ett riktigt tangentbord, eller är det utbytt mot ett som 
        sparar alla tangenttryckningar?
      \item Finns risken att det är en keylogger installerad?
    \end{itemize}
  \end{example}
\end{frame}

\subsection{Bättre lösningar?}

\begin{frame}
  \begin{itemize}
    \item Lösenord är har i sig dålig användbarhet.

    \item Finns olika metoder för att förbättra användbarheten.
      \begin{itemize}
        \item Single sign-on, exempelvis via Google eller Facebook.
        \item Spara alla lösenord krypterat och fyll automatiskt i dem på 
          webben.
          (Sabba inte detta alternativ med JavaScript.)
        \item Komplettera med koddosa.
      \end{itemize}

    \item Då har vi reducerat \(N\) lösenord till endast ett lösenord att komma 
      ihåg.

      \pause{}

    \item BankID verkar också vara en robust lösning.

  \end{itemize}
\end{frame}

\begin{frame}{BankID}
  \begin{itemize}
    \item Innebär att vi måste ha någonting: certifikatet.
    \item Vi måste veta någonting: lösenordet för certifikatet.
  \end{itemize}
\end{frame}

\begin{frame}{BankID hos Swedbank}
  % XXX get screenshots from BankID for Swedbank
  \begin{block}{Inloggning}
    I identify myself at:
    
    Swedbank och Sparbankerna
  \end{block}
  \begin{block}{Godkänna (signera) överföring}
    I sign at:
    
    Swedbank och Sparbankerna

    \vspace{1em}
    Text to be signed:

    Jag godkänner överföring med totalsumman 60,00 kr.
    Uppdraget lämnar jag till banken 2013-04-14 kl 21:25:43.
  \end{block}
\end{frame}

\begin{frame}{BankID hos Skatteverket}
  % XXX get screenshots from BankID for Skatteverket
  \begin{block}{Inloggning}
    I identify myself at:
    
    Skatteverket
  \end{block}
  \begin{block}{Signering av deklaration}
    I sign at:

    Skatteverket

    \vspace{1em}
    Text to be signed:

    Härmed undertecknar jag uppgifterna jag tidigare lämnat in.
  \end{block}
\end{frame}

\begin{frame}{BankID}
  \begin{itemize}
    \item Har separata mekanismer för identifiering och signering.
    \item Kan alltså inte lura användaren att signera en överföring vid 
      inloggning.
    \item Har ett användbart och pålitligt användargränssnitt.
%    \item Jämför med användargränssnittet till chippet på betalkortet (precis, 
%      det finns inget).
  \end{itemize}
\end{frame}

\begin{frame}
  \begin{itemize}
    \item Utforma autentisering för att användaren inte enkelt ska kunna bli 
      lurad!
    \item Om användaren blir lurad en gång ska inte det vara hela världen.
  \end{itemize}
\end{frame}

% XXX add slides on yubikey and lastpass

\begin{frame}
  \begin{itemize}
    \item Yubikey?
    \item LastPass?
  \end{itemize}
\end{frame}


\section{Take-away}

\subsection{User conditioning}

\begin{frame}
  \begin{itemize}
    \item Träna användare att använda säkra lösenord.
    \item Gör det \emph{enkelt} att använda säkra lösenord.
    \item Värdera lösenordet lika starkt som det som skyddas.
    \item Ge negativ återkoppling på dåliga lösenord.
    \item Men begär inte något de inte klarar av, då kommer policyn aldrig 
      följas.
  \end{itemize}
\end{frame}

\begin{frame}
  \begin{itemize}
    \item Marknadsföringsavdelningen vill skicka länkar.
    \item Var konsistent, ge inte användaren delade budskap.
    \item Låt dem inte göra det om ni försöker att träna användarna att använda 
      bokmärken eller skriva URL:en.
    \item Outsourceade kundundersökningar: från er (?), men med konstiga 
      URL:er:
      \enquote{There's something phishy going on.}
  \end{itemize}
\end{frame}


%%%%%%%%%%%%%%%%%%%%%%

\begin{frame}[allowframebreaks]
	\small
  \printbibliography{}
\end{frame}

\end{document}
