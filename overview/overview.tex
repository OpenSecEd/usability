%\documentclass[handout]{beamer}
\documentclass{beamer}
\usepackage[utf8]{inputenc}
\usepackage[T1]{fontenc}
\usepackage[swedish,british]{babel}
\usepackage{url}
\usepackage{graphicx}
\usepackage{color}
\usepackage{subfig}
\usepackage{multicol}
\usepackage{booktabs}
\usepackage[binary,squaren]{SIunits}
\usepackage{csquotes}

\setbeamertemplate{bibliography item}[text]
\usepackage[natbib,style=alphabetic,maxbibnames=99]{biblatex}
\addbibresource{overview.bib}

\mode<presentation>{%
  \usetheme{Berlin}
  \setbeamercovered{transparent}
}
%\setbeamertemplate{footline}{\insertframenumber}

\title{%
  Security Usability: An Overview
}
\author{Daniel Bosk}
\institute[MIUN IKS]{%
  Department of Information and Communication Systems,\\
  Mid Sweden University, SE-851\,70 Sundsvall.
}
\date{\today}

%\pgfdeclareimage[height=0.65cm]{university-logo}{MU_logotyp_int_CMYK.pdf}
%\logo{\pgfuseimage{university-logo}}

\AtBeginSection[]{%
  \begin{frame}<beamer>
    \small
    \tableofcontents[currentsection]
  \end{frame}
}

\begin{document}

\begin{frame}
  \titlepage{}
\end{frame}

\begin{frame}
  \begin{quote}
    Humans are incapable of securely storing high-quality cryptographic keys, 
    and they have unacceptable speed and accuracy when performing cryptographic 
    operations.
    (They are also large, expensive to maintain, difficult to manage, and they 
    pollute the environment.
    It is astonishing that these devices continue to be manufactured and 
    deployed.
    But they are sufficiently pervasive that we must design our protocols 
    around their limitations.)
  \end{quote}
  \begin{flushright}
    \small
    Kaufmann, Perlman and Speciner
  \end{flushright}
\end{frame}


% Since this a solution template for a generic talk, very little can
% be said about how it should be structured. However, the talk length
% of between 15min and 45min and the theme suggest that you stick to
% the following rules:  

% - Exactly two or three sections (other than the summary).
% - At *most* three subsections per section.
% - Talk about 30s to 2min per frame. So there should be between about
%   15 and 30 frames, all told.


% XXX have psychology parts connect more to security
\section{Social engineering}

\subsection{What's social engineering?}

\begin{frame}
  \begin{figure}
    \includegraphics[height=0.7\textheight]{security.png}
    \caption{''Only amateurs attack machines; professionals target people.'' 
    (Bruce Schnier).
    Bild:~\cite{xkcd538}.}
  \end{figure}
\end{frame}

\begin{frame}
  \begin{itemize}
    \item We figure out the weaknesses in human psychology.
    \item Then we exploit these to our advantage.
  \end{itemize}
\end{frame}

\subsection{Is it a problem?}

\begin{frame}
  \begin{block}{Vårt naturligt utvecklade skydd}
    \begin{itemize}
      \item Vårt naturliga skydd som utvecklats under miljontals år är baserat på 
        \enquote{här och nu}.

      \item Hotet har bytt kontext under de senaste decennierna.

      \item Evolutionen är betydligt långsammare \dots

    \end{itemize}
  \end{block}
\end{frame}

\begin{frame}
  \begin{definition}[Pretexting]
    \begin{itemize}
      \item Att ringa någon som har tillgång till informationen och låtsas vara 
        behörig att få veta.

      \item Exempelvis att låtsas vara behandlande läkare av en patient i en akut 
        situation för att få ut information ur journalen.
    \end{itemize}
  \end{definition}
\end{frame}

\begin{frame}
  \begin{example}
    \begin{itemize}
      \item Undersökning genomfördes i UK 1996~\cite{Anderson2008sea}.
      \item Utbildade personalen vid vårdinrättning om pretextingattacker.
      \item Upptäckte 30 falska samtal i veckan.
    \end{itemize}
  \end{example}
\end{frame}

\begin{frame}
  \begin{example}
    \begin{itemize}
      \item Från verkligheten~\cite{Anderson2008sea}: Ett falskt pressmeddelande 
        publicerades som sade att VD avgått och att vinsten skulle räknas om.
        \begin{itemize}
          \item Aktien föll med över \unit{60}{\%} innan det uppdagades.
        \end{itemize}

      \item Generellt går denna typ attack under \emph{social engineering}.

    \end{itemize}
  \end{example}
\end{frame}

\begin{frame}
  \begin{example}
    \begin{itemize}
      \item Vid granskning av IRS 2007 ringdes 102 personer spridda över hela 
        organisationen upp.

      \item De ombads uppge sitt användarnamn och ändra sitt lösenord till ett 
        givet värde.

      \item 62 av dem följde instruktionen.

    \end{itemize}
  \end{example}
\end{frame}

\begin{frame}
  \begin{example}
    Användare säljer sina lösenord för en chokladkaka~\cite{dn2010choklad}.
  \end{example}

  \begin{example}
    \begin{itemize}
      \item Personer tar främmande USB-minnen och använder dem med sina datorer.
        \begin{itemize}
            % XXX lookup ref for 46% of people picking up usb sticks
          \item Närmare bestämt \unit{46}{\%} av ekonomicheferna vid 500 
            börsnoterade företag~\cite{pickupusb}.
          \item \unit{66}{\%} innehåller sabotageprogram~\cite{Sophos2011usbmal}.
        \end{itemize}

    \end{itemize}
  \end{example}
\end{frame}

\subsection{Phishing}

\begin{frame}
  \begin{itemize}
    \item Militära organisationer har alltid haft varandras personal som 
      måltavla för denna typer av attacker.

    \item De har fördelen att kunna utbilda sin personal.

    \item Vanliga organisationer kan också utbilda sin personal.

    \item Det blir desto svårare att utbilda kunder eller andra personer som 
      berör verksamheten men inte är en del av organisationen.

    \item Så detta måste lösas i gränssnittet.

  \end{itemize}
\end{frame}

\begin{frame}
  \begin{itemize}
    \item Det är inte längre organisationen som angrips utan kunder och 
      personer runt omkring.

    \item Angripare återanvänder riktiga e-brev med utbytta URL:er.

    \item Vill ha ut användarnamn, lösenord, personuppgifter, \dots
  \end{itemize}

  \begin{block}{Råd}
    \begin{itemize}
      \item Klicka aldrig på URL:er som skickats till dig.
      \item Skicka aldrig klickbara \enquote{Klicka här}-URL:er till någon.
    \end{itemize}
  \end{block}

\end{frame}

\subsection{An example}

\begin{frame}
  \begin{example}[Mat Honan~\cite{MatHonan}]
    \begin{itemize}
      \item Angripare tog över och tog bort Googlekonto.
      \item Tog över Twitterkonto och postade kränkande kommentarer.
      \item Tog över AppleID-konto och tog bort alla data från alla 
        Apple-enheter.
    \end{itemize}
  \end{example}
\end{frame}

\begin{frame}
  \begin{example}[Hur?~\cite{MatHonan}]
    \begin{itemize}
      \item Brister hos Amazon.
      \item Brister hos AppleCare.
      \item Olycklig koppling av e-postadresser för Me.com och Gmail.
      \item Detta gav dem även Twitter.
    \end{itemize}
  \end{example}
\end{frame}


\section{Take-away}

\subsection{User conditioning}

\begin{frame}
  \begin{itemize}
    \item Träna användare att använda säkra lösenord.
    \item Gör det \emph{enkelt} att använda säkra lösenord.
    \item Värdera lösenordet lika starkt som det som skyddas.
    \item Ge negativ återkoppling på dåliga lösenord.
    \item Men begär inte något de inte klarar av, då kommer policyn aldrig 
      följas.
  \end{itemize}
\end{frame}

\begin{frame}
  \begin{itemize}
    \item Marknadsföringsavdelningen vill skicka länkar.
    \item Var konsistent, ge inte användaren delade budskap.
    \item Låt dem inte göra det om ni försöker att träna användarna att använda 
      bokmärken eller skriva URL:en.
    \item Outsourceade kundundersökningar: från er (?), men med konstiga 
      URL:er:
      \enquote{There's something phishy going on.}
  \end{itemize}
\end{frame}


%%%%%%%%%%%%%%%%%%%%%%

\begin{frame}[allowframebreaks]
	\small
  \printbibliography{}
\end{frame}

\end{document}
