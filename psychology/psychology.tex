%\documentclass[handout]{beamer}
\documentclass{beamer}
\usepackage[utf8]{inputenc}
\usepackage[T1]{fontenc}
\usepackage[swedish,british]{babel}
\usepackage{url}
\usepackage{graphicx}
\usepackage{color}
\usepackage{subfig}
\usepackage{multicol}
\usepackage{booktabs}
\usepackage{csquotes}
\usepackage[binary-units]{siunitx}

\usepackage[all]{foreign}
\renewcommand{\foreignabbrfont}{\relax}

\usepackage[natbib,style=alphabetic,maxbibnames=99]{biblatex}
\addbibresource{bibliography.bib}


\mode<presentation>{%
  \usetheme{Berlin}
  \setbeamercovered{transparent}
}
%\setbeamertemplate{footline}{\insertframenumber}

\title{%
  Psychology in security
}
\author{Daniel Bosk}
\institute[MIUN IKS]{%
  Department of Information and Communication Systems,\\
  Mid Sweden University, SE-851\,70 Sundsvall.
}
\date{\today}

%\pgfdeclareimage[height=0.65cm]{university-logo}{MU_logotyp_int_CMYK.pdf}
%\logo{\pgfuseimage{university-logo}}

\AtBeginSection[]{%
  \begin{frame}<beamer>
    \small
    \tableofcontents[currentsection]
  \end{frame}
}

\begin{document}

\begin{frame}
  \titlepage{}
\end{frame}

%\begin{frame}
%  \begin{quote}
%    Humans are incapable of securely storing high-quality cryptographic keys, 
%    and they have unacceptable speed and accuracy when performing cryptographic 
%    operations.
%    (They are also large, expensive to maintain, difficult to manage, and they 
%    pollute the environment.
%    It is astonishing that these devices continue to be manufactured and 
%    deployed.
%    But they are sufficiently pervasive that we must design our protocols 
%    around their limitations.)
%  \end{quote}
%  \begin{flushright}
%    \small
%    Kaufmann, Perlman and Speciner
%  \end{flushright}
%\end{frame}

\mode<all>{\mode*

% Since this a solution template for a generic talk, very little can
% be said about how it should be structured. However, the talk length
% of between 15min and 45min and the theme suggest that you stick to
% the following rules:  

% - Exactly two or three sections (other than the summary).
% - At *most* three subsections per section.
% - Talk about 30s to 2min per frame. So there should be between about
%   15 and 30 frames, all told.


\section{How humans function}

\subsection{Some basic concepts}

\begin{frame}
  \begin{definition}[Mental models]
    \begin{itemize}
      \item Mental models allows us to reason about the real world.
      \item They're simplifications of real world items.
    \end{itemize}
  \end{definition}

  \pause

  \begin{example}
    \begin{itemize}
      \item Water tap: the more you open, the more water flows.
      \item Thermostat: the higher temperature you put, the faster the place 
        heats up?
      \item Stove: increase the setting, heat up faster.
    \end{itemize}
  \end{example}
\end{frame}

\begin{frame}
  \begin{example}[Programming]
    \begin{itemize}
      \item The integers are infinite.
      \item The computer does everything modulo \(2^{32}\) or \(2^{64}\).
    \end{itemize}
  \end{example}
\end{frame}

\begin{frame}
  \begin{example}[Secure connection]
    \begin{itemize}
      \item Users don't understand SSL/TLS\@.
      \item The padlock in the browser means we're secure?
      \item A phishing page has a signed certificate \dots
    \end{itemize}
  \end{example}

  \pause

  \begin{example}
    \begin{itemize}
      \item Users don't know which part of the interface is trustworthy.
      \item The webpage shows a green padlock \dots
    \end{itemize}
  \end{example}
\end{frame}

\begin{frame}
  \begin{definition}[Capture errors]
    A trained behaviour is used instead of the correct one.
  \end{definition}

  \pause

  \begin{example}
    \begin{itemize}
      \item I drive to work everyday.
      \item The few times I'm supposed to drive the other direction \dots
      \item I still drive towards work and must turn.
      \item I go home instead of to the store.
      \item Users automatically click the OK button without thinking \dots
    \end{itemize}
  \end{example}
\end{frame}

\begin{frame}
  \begin{block}{Post-completion error}
    When the goal is reached, the task is complete.
    Or not.
  \end{block}

  \pause

  \begin{example}
    \begin{itemize}
      \item There's a reason ATMs return your card before the cash.
      \item Otherwise people will forget their cards in the ATMs.
    \end{itemize}
  \end{example}
\end{frame}

\begin{frame}
  \begin{block}{Cognitive load}
    \begin{itemize}
      \item Cognitive load affects us a lot.
      \item Stress, multitasking, \etc.
      \item The \enquote{autopilot} kicks in.
    \end{itemize}
  \end{block}
  
  \pause

  \begin{example}
    \begin{itemize}
      \item Most times you drive the wrong way (capture error), you've
      \item been thinking deeply about something,
      \item been engaged in a conversation with someone,
      \item \etc.
    \end{itemize}
  \end{example}
\end{frame}

\begin{frame}
  \begin{figure}
    \includegraphics[height=0.7\textheight]{citations.png}
    \caption{A short comic on illustrates these types of effects.
      Image:~\cite{xkcd906}.}
  \end{figure}
\end{frame}

\subsection{Biases}

\begin{frame}
  \begin{block}{Automation bias}
    We trust the computer to have done the work properly.
  \end{block}

  \begin{remark}
    \begin{itemize}
      \item The computer is perceived as objective.
      \item We don't question it.
      \item But the computer is \emph{not objective}.
      \item These algorithms are also biased.
    \end{itemize}
  \end{remark}
\end{frame}

\begin{frame}
  \begin{example}
    \begin{itemize}
      \item Two studies looked at how people use Google results.

      \item Students picked links higher up although the abstracts were less 
        relevant~\cite{Pan2007igw}.

      \item The researchers could affect the voting preferences by changing the 
        search results~\cite{Epstein2013dar}.

      \item The subjects were not aware of this change \dots
    \end{itemize}
  \end{example}
\end{frame}

\begin{frame}
  \begin{block}{Confirmation bias}
    We seek out information that confirms our beliefs.
  \end{block}

  \pause

  \begin{example}
    \begin{itemize}
      \item This makes us bad at testing hypotheses.
      \item Test if the site is a phishing site by giving it username and 
        password.
      \item If the site knows them, it must be legit.
    \end{itemize}
  \end{example}

  \pause

  \begin{remark}[Reverse-authorization fallacy]
    Give me the username and password and I'll verify them.
  \end{remark}
\end{frame}

\begin{frame}
  \begin{block}{Disconfirmation bias}
    We rather accept plausible but wrong, than implausible but correct.
  \end{block}

  \pause

  \begin{example}
    \begin{itemize}
      \item If the website behaves as the bank \dots
      \item Then it must be the bank \dots
      \item Although it's run from Russia.
    \end{itemize}
  \end{example}
\end{frame}

\begin{frame}
  \begin{block}{Projection bias}
    Everyone thinks like me.
  \end{block}

  \begin{example}
    \begin{itemize}
      \item Someone designs a system and expects the users to think like the 
        designer.
      \item If you're logged in, then you must be a good guy like me.
    \end{itemize}
  \end{example}
\end{frame}

\begin{frame}
  \begin{block}{Blind-spot bias}
    Biases are unconscious, that makes them difficult to see for conscious 
    thought.
  \end{block}
\end{frame}

%\subsection{How me make decisions}
%
%\begin{frame}
%  \begin{itemize}
%    \item Heuristiker människor använder för beslut ligger på gränsen mellan 
%      rationellt tänkande och direkta sinnesintryck.
%
%    \item Risker: I många fall tycker vi mindre om att förlora \unit{100}{kr} 
%      vi redan har än att vinna \unit{100}{kr} vi inte har.
%      \begin{itemize}
%        \item Marknadsföring: ''spara \unit{100}{kr}''.
%      \end{itemize}
%
%    \item Vi är dåliga på att uppskatta sannolikheter:
%      \begin{itemize}
%        \item Baserar härledningar på enkla analogier.
%        \item Tillgänglighetsheuristiken: lättillgängliga data har större vikt 
%          vid resonemang.
%        \item Och vi jämför med nyliga händelser.
%        \item Förankringseffekten: vi gör en initial uppskattning och 
%          förbättrar vid behov.
%      \end{itemize}
%
%  \end{itemize}
%\end{frame}
%
%\begin{frame}
%  \begin{itemize}
%    \item Vi är mer benägna att vara skeptiska mot något vi hört än något vi 
%      sett.
%    \item Överskattar risken för terroristattacker jämfört med bilolyckor.
%    \item Beror dels på synligheten i media.
%    \item Gilbert: ''If only gay sex caused global warming''.
%    \item Global uppvärmning bryter inte mot någons (religiösa?) värderingar, 
%      är långtgående hot.
%    \item Vi är anpassade för snabba förändringar i omvärlden och tydliga akuta 
%      hot.
%    \item Vi är också mindre rädda när vi (tror) att vi har kontroll, 
%      exempelvis att köra bilen jämfört med att sitta som passagerare.
%    \item Vi är också riskaversiva: ''kvitt eller dubbelt'' går inte hem.
%  \end{itemize}
%\end{frame}
%
%\begin{frame}
%  \begin{itemize}
%    \item Kan dela upp psyket i kognitivt och affektivt system.
%
%    \item Utvecklingsbiologin har sett att olika processer används för sociala 
%      och fysikaliska fenomen.
%
%    \item Barn försöker att förklara händelser med sin förståelse för fysik, 
%      när den inte räcker försöker de förklara med avsiktliga handlingar (det 
%      affektiva tar vid).
%
%    \item Detta leder till att de tar hjälp av vuxna.
%
%  \end{itemize}
%\end{frame}
%
%\begin{frame}{Bieffekter}
%  \begin{itemize}
%    \item Människor försöker att förklara händelser med intention snarare än 
%      situation.
%
%    \item Vidare leder detta till att om vi får den affektiva sidan att ta över 
%      är personen mindre uppmärksam och sämre att uppskatta sannolikheter.
%
%  \end{itemize}
%\end{frame}

%\subsection{Socialpsykologi}
%
%\begin{frame}
%  \begin{itemize}
%    \item Det sociala samspelet har stor inverkan på individen.
%
%    \item 1951 visades att en individ kunde bortse från uppenbara bevis bara 
%      för att följa gruppen.
%
%      \pause{}
%
%    \item Vidare har visats att individer kan göra helt moralvidriga saker 
%      under order från en auktoritet, Officer Scott 1995--2005:
%      \begin{itemize}
%        \item Ringde upp restaurangchefer och låtsades vara polis.
%        \item Tvingade fram strippsökningar av oskyldiga unga anställda.
%      \end{itemize}
%
%      \pause{}
%
%    \item Detta kan ske även utan order från en auktoritet, Stanford Prisoner 
%      Experiment 1971:
%      \begin{itemize}
%        \item 12 vakter, 12 fångar.
%        \item Vakterna blev snabbt sadistiska auktoriteter.
%      \end{itemize}
%
%  \end{itemize}
%\end{frame}
%
%\begin{frame}
%  \begin{itemize}
%    \item Sociala medier?
%
%      \pause{}
%
%    \item Det ser ut som att \enquote{alla andra} har gjort det.
%    \item Exempelvis bedrägerier som sprids via Facebook.
%    \item Oftast är de orsakade av sabotageprogram.
%  \end{itemize}
%\end{frame}

%\begin{frame}{Kognitiv dissonansteori}
%  \begin{itemize}
%    \item Människor tycker inte om motstridigheter.
%
%    \item Söker information för att bekräfta tidigare kunskap eller mentala 
%      modeller.
%
%    \item Information som motstrider bortses ifrån, vill inte tro att man 
%      tidigare haft fel eller bryta från alla andra.
%
%  \end{itemize}
%\end{frame}



%%%%%%%%%%%%%%%%%%%%%%

\begin{frame}[allowframebreaks]
	\small
  \printbibliography
\end{frame}

}
\end{document}
