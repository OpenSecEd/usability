% Since this a solution template for a generic talk, very little can
% be said about how it should be structured. However, the talk length
% of between 15min and 45min and the theme suggest that you stick to
% the following rules:  

% - Exactly two or three sections (other than the summary).
% - At *most* three subsections per section.
% - Talk about 30s to 2min per frame. So there should be between about
%   15 and 30 frames, all told.


\section{How humans function}

\subsection{Grundläggande psykologi}

\begin{frame}
  \begin{block}{Mental models}
    \begin{itemize}
      \item Mentala modeller låter oss identifiera människor, ljud, 
        \enquote{koncept} bättre än datorer.

      \item Dock gör oss även sårbara när fel modell aktiveras eller modellen 
        inte är i linje med verkligheten.
    \end{itemize}
  \end{block}

  \pause{}

  \begin{example}[Säker anslutning]
    \begin{itemize}
      \item Användaren förstår inte SSL/TLS\@.
      \item Hänglåset betyder säkerhet.
      \item En phishingsida har ett signerat certifikat.
    \end{itemize}
  \end{example}
\end{frame}

\begin{frame}
  \begin{block}{Capture errors}
    Ett inövat beteende används istället för korrekt.
  \end{block}

  \pause{}

  \begin{example}
    \begin{itemize}
      \item Svänger ut på motorvägen mot Sundsvall istället för Härnösand.
      \item Åker hem istället för till affären efter jobbet.
      \item Klickar \enquote{automatiskt} på OK-knappen utan att tänka efter.
    \end{itemize}
  \end{example}
\end{frame}

\begin{frame}
  \begin{block}{Post-completion error}
    När målet är nått är uppgiften genomförd, eller \dots
  \end{block}

  \pause{}

  \begin{example}
    \begin{itemize}
      \item Uttagsautomater som ger pengarna före kortet gör att fler glömmer 
        kortet i automaten.
    \end{itemize}
  \end{example}
\end{frame}

\begin{frame}
  \begin{figure}
    \includegraphics[height=0.7\textheight]{citations.png}
    \caption{En kort seriestrip om hur vi är okritiska mot text som, till 
    synes, har referenser.
    Bild:~\cite{xkcd906}.}
  \end{figure}
\end{frame}

\begin{frame}
  \begin{block}{Cognitive load}
    \begin{itemize}
      \item Kognitiv belastning påverkar oss hårt.
      \item Handlingar som följer någon form av regel.
      \item Vid hög kognitiv belastning kan fel regel följas, exempelvis 
        starkaste regeln istället för lämpligaste.
    \end{itemize}
  \end{block}
  
  \pause{}

  \begin{example}
    \begin{itemize}
      \item Det är säkert då det står \enquote{https} i URL:en, eller ikonen 
        med hänglåset.
      \item Att hitta bankens namn är en starkare regel än att tänka på dess 
        position; \url{https://www.swedbank.se.fraudulentbanks.com}.
    \end{itemize}
  \end{example}
\end{frame}

\subsection{Biases}

\begin{frame}
  \begin{block}{Automation bias}
    We trust the computer to have done the work properly, so we relax.
  \end{block}
  \begin{example}
    \begin{itemize}
      \item Två studier visar att tilliten till resultaten från Google är stor.

      \item Studenter valde länkar högre upp i träfflistan trots att 
        sammanfattningarna var mindre relevanta än träffar längre 
        ned~\cite{Pan2007igw}.

      \item I en nyligare genomförd studie~\cite{Epstein2013dar} visas att 
        sökresultat kan förändra personers röstningspreferenser utan att 
        uppmärksammas.

      \item Detta gör sökmotoroptimering till ett farligt område.

    \end{itemize}
  \end{example}
\end{frame}

\begin{frame}
  \begin{block}{Confirmation bias}
    We seek out information that confirms our beliefs.
  \end{block}

  \pause{}

  \begin{example}
    \begin{itemize}
      \item This makes us bad at testing hypotheses.
      \item Test if the site is a phishing site by giving it username and 
        password.
      \item If the site knows them, it must be legit.
    \end{itemize}
  \end{example}

  \pause{}

  \begin{block}{Reverse-authorization fallacy}
    Give me the username and password and I'll verify them.
  \end{block}
\end{frame}

\begin{frame}
  \begin{block}{Disconfirmation bias}
    We rather accept plausible but wrong, than implausible but correct.
  \end{block}

  \pause{}

  \begin{example}
    \begin{itemize}
      \item If the website behaves as the bank \dots
      \item Then it must be the bank \dots
      \item Although it's run from Russia.
    \end{itemize}
  \end{example}
\end{frame}

\begin{frame}
  \begin{block}{Projection bias}
    Everyone thinks like me.
  \end{block}

  \begin{example}
    \begin{itemize}
      \item Someone designs a system and expects the users to think like the 
        designer.
      \item If you're logged in, then you must be a good guy like me.
    \end{itemize}
  \end{example}
\end{frame}

\begin{frame}
  \begin{block}{Blind-spot bias}
    Biases are unconscious, that makes them difficult to see for conscious 
    thought.
  \end{block}
\end{frame}

%\subsection{How me make decisions}
%
%\begin{frame}
%  \begin{itemize}
%    \item Heuristiker människor använder för beslut ligger på gränsen mellan 
%      rationellt tänkande och direkta sinnesintryck.
%
%    \item Risker: I många fall tycker vi mindre om att förlora \unit{100}{kr} 
%      vi redan har än att vinna \unit{100}{kr} vi inte har.
%      \begin{itemize}
%        \item Marknadsföring: ''spara \unit{100}{kr}''.
%      \end{itemize}
%
%    \item Vi är dåliga på att uppskatta sannolikheter:
%      \begin{itemize}
%        \item Baserar härledningar på enkla analogier.
%        \item Tillgänglighetsheuristiken: lättillgängliga data har större vikt 
%          vid resonemang.
%        \item Och vi jämför med nyliga händelser.
%        \item Förankringseffekten: vi gör en initial uppskattning och 
%          förbättrar vid behov.
%      \end{itemize}
%
%  \end{itemize}
%\end{frame}
%
%\begin{frame}
%  \begin{itemize}
%    \item Vi är mer benägna att vara skeptiska mot något vi hört än något vi 
%      sett.
%    \item Överskattar risken för terroristattacker jämfört med bilolyckor.
%    \item Beror dels på synligheten i media.
%    \item Gilbert: ''If only gay sex caused global warming''.
%    \item Global uppvärmning bryter inte mot någons (religiösa?) värderingar, 
%      är långtgående hot.
%    \item Vi är anpassade för snabba förändringar i omvärlden och tydliga akuta 
%      hot.
%    \item Vi är också mindre rädda när vi (tror) att vi har kontroll, 
%      exempelvis att köra bilen jämfört med att sitta som passagerare.
%    \item Vi är också riskaversiva: ''kvitt eller dubbelt'' går inte hem.
%  \end{itemize}
%\end{frame}
%
%\begin{frame}
%  \begin{itemize}
%    \item Kan dela upp psyket i kognitivt och affektivt system.
%
%    \item Utvecklingsbiologin har sett att olika processer används för sociala 
%      och fysikaliska fenomen.
%
%    \item Barn försöker att förklara händelser med sin förståelse för fysik, 
%      när den inte räcker försöker de förklara med avsiktliga handlingar (det 
%      affektiva tar vid).
%
%    \item Detta leder till att de tar hjälp av vuxna.
%
%  \end{itemize}
%\end{frame}
%
%\begin{frame}{Bieffekter}
%  \begin{itemize}
%    \item Människor försöker att förklara händelser med intention snarare än 
%      situation.
%
%    \item Vidare leder detta till att om vi får den affektiva sidan att ta över 
%      är personen mindre uppmärksam och sämre att uppskatta sannolikheter.
%
%  \end{itemize}
%\end{frame}

\subsection{Socialpsykologi}

\begin{frame}
  \begin{itemize}
    \item Det sociala samspelet har stor inverkan på individen.

    \item 1951 visades att en individ kunde bortse från uppenbara bevis bara 
      för att följa gruppen.

      \pause{}

    \item Vidare har visats att individer kan göra helt moralvidriga saker 
      under order från en auktoritet, Officer Scott 1995--2005:
      \begin{itemize}
        \item Ringde upp restaurangchefer och låtsades vara polis.
        \item Tvingade fram strippsökningar av oskyldiga unga anställda.
      \end{itemize}

      \pause{}

    \item Detta kan ske även utan order från en auktoritet, Stanford Prisoner 
      Experiment 1971:
      \begin{itemize}
        \item 12 vakter, 12 fångar.
        \item Vakterna blev snabbt sadistiska auktoriteter.
      \end{itemize}

  \end{itemize}
\end{frame}

\begin{frame}
  \begin{itemize}
    \item Sociala medier?

      \pause{}

    \item Det ser ut som att \enquote{alla andra} har gjort det.
    \item Exempelvis bedrägerier som sprids via Facebook.
    \item Oftast är de orsakade av sabotageprogram.
  \end{itemize}
\end{frame}

%\begin{frame}{Kognitiv dissonansteori}
%  \begin{itemize}
%    \item Människor tycker inte om motstridigheter.
%
%    \item Söker information för att bekräfta tidigare kunskap eller mentala 
%      modeller.
%
%    \item Information som motstrider bortses ifrån, vill inte tro att man 
%      tidigare haft fel eller bryta från alla andra.
%
%  \end{itemize}
%\end{frame}



%%%%%%%%%%%%%%%%%%%%%%

\begin{frame}[allowframebreaks]
	\small
  \printbibliography
\end{frame}

